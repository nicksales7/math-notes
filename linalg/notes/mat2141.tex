\documentclass[11pt]{article}

% Packages
\usepackage[utf8]{inputenc}
\usepackage{geometry}
\usepackage{amsmath}
\usepackage{amsfonts}
\usepackage{amssymb}
\usepackage{graphicx}
\usepackage{hyperref}
\usepackage{import}
\usepackage{xifthen}
\usepackage{pdfpages}
\usepackage{transparent}
\usepackage{tikz}
\usepackage{pgfplots}
\pgfplotsset{compat=newest}

\newcommand{\incfig}[1]{%
    \def\svgwidth{8cm}
    \import{./figures/}{#1.pdf_tex}
}

% Document settings
\geometry{a4paper, margin=1in}
\setlength\parindent{0pt}
\setlength\parskip{1em}

\title{MAT 2141 Notes}
\author{Nicholas Sales}
\date{} % This will automatically insert today's date

\begin{document}

\maketitle

\tableofcontents
\newpage

\section{Linear Maps}

\subsection{Definiton}

\textbf{Definition 1.1.1:} If $V$ and $W$ are vector spaces over the same field $F$, and $T : V \rightarrow W$ is a mapping, we say $T$ is a linear mapping if the following hold:
\begin{gather*}
    T(u + v) = T(u) + T(v) \;\; \forall u,v \in V \\
    T(cv) = cT(v) \;\; \forall v \in V, \; \forall c \in F
\end{gather*}
\textbf{Example 1.1.1:} Let $V = C^{\infty}(\mathbb{R}), \; F = \mathbb{R}$, and define $S : C^{\infty}(\mathbb{R}) \rightarrow C^{\infty}(\mathbb{R})$ by $(Sf)(x) = \int_{0}^{x} f(t) dt, \; \forall x \in \mathbb{R}$. \\
\vspace{0.1cm} \\
Let $f, g \in V$ (i.e. $f$ and $g$ are infinitely differentiable functions) and let $c \in F$. Using the properties of integration, we can show additivity and homogeneity hold:
\begin{gather*}
    S(f + g)(x) = \int_{0}^{x} f(t) + g(t) dt = \int_{0}^{x} f(t) dt + \int_{0}^{x} g(t) dt = (Sf)(x) + (Sg)(x) \\
    S(cf)(x) = \int_{0}^{x} cf(t) dt = c \int_{0}^{x} f(t) dt = c(Sf)(x)
\end{gather*}
Since $S$ satisfies the properties of additivity and homogeneity, we conclude $S$ is a linear map. QED.

\subsection{Kernel and Image}

\textbf{Definiton 1.2.1:} If $T : V \rightarrow W$ is a linear map, then $Ker(T) \subseteq V$ gets mapped to the zero vector under $T$ and $Im(T) \subseteq W$ is where vectors from $V$ can actually get mapped to under $T$ (i.e. range):
\begin{gather*}
    Ker(T) = T^{-1}(\vec{0}) = \{x \; | \; Tx = \vec{0}\} \\
    Im(T) = T(V) = \{Tx \; | \; x \in V\}
\end{gather*}
\textbf{Theorem 1.2.1:} Suppose $T : V \rightarrow W$ is a linear mapping, $T$ is injective iff $Ker(T) = \{\vec{0}\}$. \\ 
\vspace{0.1cm} \\
We prove this in two directions. Assume $T$ is injective and let $u,v \in V$. Since $0v = \vec{0}$ for any vector $v$ and $T(0v) = 0T(v)$ by the property of linear maps, it follows a linear mapping always maps the zero vector to the zero vector. Since for any injective mapping we have $T(u) = T(v) \implies u = v$, it follows if $T(u) = T(v) = \vec{0}$, then $u = v = \vec{0}$ and thus $Ker(T) = \{\vec{0}\}$. \\
\vspace{0.1cm} \\
We now prove this in the other direction. Assume $Ker(T) = \vec{0}$. We want to show if $T(u) = T(v)$ for any $u,v \in V$ then $u = v$. If $T(u) = T(v)$, it follows $T(u) - T(v) = \vec{0}$ which simplifies to $T(u - v) = \vec{0}$ by the property of linear maps. Since $Ker(T) = \{\vec{0}\}$ it follows that $T(u - v) = \vec{0} \implies u - v = \vec{0} \implies u = v$. Thus $T$ is injective. Since we have proved both implications this completes the proof. QED.

\newpage

\subsection{Vector Spaces of Linear Maps}

\textbf{Definition 1.3.1:} If $V, W$ are vector spaces over a field $F$, we define the set of linear maps from $V$ to $W$:
\begin{equation*}
    \mathcal{L}(V,W) = \{T : V \rightarrow W \; | \; T \text{ is linear}\}
\end{equation*}
If $V = W$, we simply write $\mathcal{L}(V)$. \\
\textbf{Definition 1.3.2:} If $A, B,$ and $C$ are sets with mappings $T : A \rightarrow B$ and $S : B \rightarrow C$, we can compose the mappings to get $ST : A \rightarrow C$ (i.e. $S \circ T$) defined by:
\begin{equation*}
    (ST)a = S(Ta) \;\; \forall a \in A
\end{equation*}
\textbf{Theorem 1.3.1:} The composition of linear maps is itself a linear map (i.e. $T \in \mathcal{L}(U,V)$ and $S \in \mathcal{L}(V,W) \implies ST \in \mathcal{L}(U,W)$). \\
\vspace{0.1cm} \\
Let $T \in \mathcal{L}(U,V)$ and $S \in \mathcal{L}(V,W)$ be linear maps over a field $F$ with $u,v \in U$ and $c \in F$. By the definition of composition we have:
\begin{gather*}
    (ST)(u + v) = S(T(u + v)) \\
    (ST)(cu) =  S(T(cu))
\end{gather*}
By the property of $T$ being a linear map we get:
\begin{gather*}
    S(T(u + v)) = S(T(u) + T(v)) \\
    S(T(cu)) = S(cT(u))
\end{gather*}
And further, by the property of $S$ being a linear map we get:
\begin{gather*}
    S(T(u) + T(v)) = S(T(u)) + S(T(v)) \\
    S(cT(u)) = cS(T(u))
\end{gather*}
Thus, $ST$ satisfies additivity and homogeneity and is therefore a linear map. QED.

\subsection{Isomorphisms}

\textbf{Definition 1.4.1:} If $T : V \rightarrow W$ is a bijective linear map, we say $T$ is an isomorphism. We also say that $V$ is isomorphic to $W$ and write $V \cong W$. \\
\textbf{Theorem 1.4.1:} Suppose $V$ and $W$ are vector spaces. Further, assume $T : V \rightarrow W$ is an isomorphism. It follows that $T^{-1} : W \rightarrow V$ is also linear and hence an isomorphism. \\
\vspace{0.1cm} \\
Assume $T : V \rightarrow W$ is an isomorphism. Since it follows that $T$ is bijective, it is invertible. Let $u',v' \in W$. It follows there exists some $u,v \in V$ such that $T(u) = u'$ and $T(v) = v'$. By the property of $T$ being linear, we have:
\begin{gather*}
    T^{-1}(u' + v') = T^{-1}(T(u) + T(v)) = T^{-1}(T(u+v)) = u + v = T^{-1}(u') + T^{-1}(v')\\
    T^{-1}(cu') = T^{-1}(cT(u)) = T^{-1}(T(cu)) = cu = cT^{-1}(u')
\end{gather*}
Thus $T^{-1}$ preserves additivity and homogeneity and is therefore a linear map. And moreover, since it is the inverse of a bijective mapping, it is also bijective, and thus an isomorphism. QED. 

\newpage

\textbf{Theorem 1.4.2:} Isomorphisms are an equivalence relation. In other words, if $U,V,W$ are all vector spaces over the same field $F$, the following properties hold:
\begin{gather*}
    U \cong U \\ 
    U \cong V \implies V \cong U \\
    U \cong V \text{ and } V \cong W \implies U \cong W
\end{gather*}
This proof is left to the non-existent reader.

\section{Structure of Vector Spaces}

\subsection{Spans and Generating Sets}


\end{document}
